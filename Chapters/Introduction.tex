\chapter{Introduction}
\label{chap:introduction}

\section{Topic covered by the project}

\section{Keywords}

\section{Problem description}
\label{sec:ProblemDescription}
Several games use random number generators to procedurally generate puzzles, levels, worlds, or universes~\cite{Hendrikx2013}.
Some of these may produce near infinite worlds, only limited by the discrete properties of a computer.
Minecraft is such a game, where an infinite world of cubes is generated~\cite{Minecraft:Website}, and No Man's Sky is game that generates a universe with galaxies, star systems, planets, ships, animals and plants~\cite{NMS:Website}.
With infinite possibilities, there is a problem of generating interesting levels.

A simple example is Sudoku, where a user may want to try a challenging puzzle.
A computer algorithm can generate a random puzzle, but it does not know if the puzzle will be challenging for the user.
A human can assess the level of difficulty in the puzzle, but this is not viable if thousands, or infinite number of puzzles are generated.
The same applies to a game like No Man's Sky where 18 quintillion planets are generated~\cite{NMS:Size}.
A human player may stop playing if the planets are not interesting. Therefore, a way of assessing a generated level without human interaction is needed.


%What’s ’wrong’ with the world we’re living in? E.g.
%• Something is currently too difficult.
%• Something is broken/doesn’t work properly.
%• Something is currently to expensive, difficult, costly etc.
%Control questions:
%1. Does it have an appropriate length?
%2. Would it be possible to explain the problem description to a non-expert/expert in say
%2 minutes in such a way that it was understood?
%3. If explained to different people, would they have a common understanding?
%4. If you were to check if your problem description was understood, what question(s) would you ask?
%5. What is the information density of your text and why?




\section{Justification, motivation and benefit}
%Visualisation has been suggested to be a valuable tool to disseminate big data. As the basic premise goes, visualisation of big data could help people to spot meanings that machines are not able to notice, to detect hidden patterns and correlations - Immersive and Collaborative Data Visualization Using Virtual Reality Platforms (Article) 
\section{Research question}

\section{Planned Contribution}

