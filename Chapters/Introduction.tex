\chapter{Introduction}
\label{chap:introduction}

\section{Topic covered by the project}

\section{Keywords}

\section{Problem description}
\label{sec:ProblemDescription}


  

%What’s ’wrong’ with the world we’re living in? E.g.
%• Something is currently too difficult.
%• Something is broken/doesn’t work properly.
%• Something is currently to expensive, difficult, costly etc.
%Control questions:
%1. Does it have an appropriate length?
%2. Would it be possible to explain the problem description to a non-expert/expert in say
%2 minutes in such a way that it was understood?
%3. If explained to different people, would they have a common understanding?
%4. If you were to check if your problem description was understood, what question(s) would you ask?
%5. What is the information density of your text and why?




\section{Justification, motivation and benefit}
%Visualisation has been suggested to be a valuable tool to disseminate big data. As the basic premise goes, visualisation of big data could help people to spot meanings that machines are not able to notice, to detect hidden patterns and correlations - Immersive and Collaborative Data Visualization Using Virtual Reality Platforms (Article) 
\section{Research question}

\section{Planned Contribution}

